\section{Multi-Party Computation (MPC)}

The second approach to ensuring trust in the Sphinx header is to prevent user manipulation by decentralizing the scheme through the use of Multi-Party Computaftion (MPC).


One major challenge in decentralizing the current schema (Figure \ref{fig:header_cipher}) lies in the integrity tag, which is the HMAC of the header $\beta_i$ computed using the shared secret $s_i$. To ensure security, no third party should have access to $s_i$, as collusion among third parties could lead to the recovery of all $s_i$, enabling them to decrypt the header and payload.

To address this, third parties could compute a portion of the HMAC using a fragment of the shared secret $s_i$, and then combine these partial HMACs to produce the final HMAC. This approach would require a hash function with homomorphic properties, which inherently weakens the collusion resistance and potentially compromises the second pre-image resistance of a secure hash.

However, even if breaking this homomorphic hash is feasible, if it remains computationally hard enough (e.g., requiring several hours), it could still be considered sufficiently secure for our purposes.